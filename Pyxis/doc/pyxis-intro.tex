\documentclass[a4paper,10pt]{article}
\usepackage[utf8]{inputenc}
\usepackage{verbatim}

\topmargin=.5cm
\oddsidemargin=-.5cm
\evensidemargin=-.5cm
\textwidth=18cm

%opening
\title{An Introduction To Pyxis\\
(Python eXtensions For Interferometry Scripting)}
\author{O.M. Smirnov (Rhodes University \& SKA SA)\\
{\tt oms@ska.ac.za}}

\begin{document}

\maketitle

\section{Introduction}

Pyxis is a set of Python-based scripting services that is meant to simplify the process of making data reduction and simulation scripts, using MeqTrees and other diverse bits of software. Pyxis scripts are essentially Python scripts, with extra convenience features provided by the Pyxis framework\footnote{Pyxis borrows interface ideas from, among others, the Perl language, casapy, pybdsm, etc.}. Pyxis scripts can be run from the command line, or interactively via ipython.

\subsection{Quick start}

Make a new subdirectory, and copy into it the files {\tt pyxis-wsrt21cm.py}, {\tt pyxis-A773.conf}, {\tt calico-stefcal.py} and {\tt tdlconf.profiles} from {\tt Pyxis-A773-Example/}. Edit {\tt pyxis-A773.conf} to make sure that {\tt MS\_TARBALL\_DIR} is set to the full path to {\tt Pyxis-A773-Example/data}; that directory should contain eight MS tarballs ({\tt A773\_B0} to {\tt A773\_B7.MS.tgz}).

Run

\begin{verbatim}
  $ pyxis A773_B0.MS reset_ms cal_ms
\end{verbatim}

to calibrate one MS. This will take 10--20 minutes. At the end of it, various output files and FITS images will appear under
{\tt plots-A773\_B0}\footnote{Try ``{\tt ls -lrt plots-A773\_B0/}'' to get an idea of the sequence in which things are generated.}. If all goes well, try

\begin{verbatim}
  $ pyxis -j8 runall
\end{verbatim}

...to process all 8 MSs at once (if you have fewer than 8 cores, e.g. 2, use {\tt -j2}. Omit the {\tt -j} argument entirely 
to run through the MSs one by one.) The final output image should appear in {\tt mean-fullrest8.fits}.

\subsection{Recipes, configs, modules} 

This is how a typical Pyxis workflow is structured:

\begin{itemize}
  \item Each ``project'' (i.e. each data reduction, each simulations project, etc.) should live in its own directory.
  
  \item The project directory contains one or more Pyxis {\em recipes} (named {\tt pyxis-*.py}), and 
  optionally some {\em configs} (named {\tt pyxis-*.conf}). These are loaded by Pyxis {\bf automatically} whenever it is run.
  
  \item Recipes are bits of Python (collections of Python functions) implementing some kind of data reduction procedure. These are meant to be somewhat application-specific. In the example here, we look at a recipe for some WSRT 21cm selfcal.
  
  \item Configs are also Python code; they are meant to augment recipes by setting up project-specific variables. Configs are totally optional -- it's perfectly possible to define all the relevant variables directly inside the recipe file instead -- but ``good practice'' is to keep the more general procedures in the recipe, and the specific stuff (MS names, sky models, etc.) in the config. This makes it easier to share recipes.

  \item Somewhere under your {\tt \$PYTHONPATH} is a directory called {\tt Pyxides}; this contains a number of {\em modules} 
  that are the real building blocks of recipes. Pyxides modules are meant to implement more generic services, and interfaces to external software. Current modules include:

  \begin{description}
    \item[ms:] interface to the MS and various simple MS-related tasks.
    \item[imager:] interface to the imager (currently supports {\tt lwimager}, but others could easily be added)  
    \item[mqt:] a generic interface to MeqTrees batch mode
    \item[cal:] various calibration-related tasks (e.g. one step of Stefcal, one run of PyBDSM, etc.)
\end{description}
  
  \item The front-end to Pyxis itself is provided by the {\tt pyxis} executable. There is also a 
  {\tt Pyxis} module for interactive use with e.g. ipython. These reside in {\tt Pyxis/}, 
  somewhere under your {\tt \$PYTHONPATH}.

\end{itemize}

\subsection{A case study}

Appendices~\ref{app:recipe} and \ref{app:config} provide a recipe for WSRT 21cm selfcal, and 
a config file for a specific bunch of observations. These are the same files you used in the quick-start section above.
You're encouraged to give them a quick browse before you read on.

Running Pyxis is very simple. Given the setup outlined in the  appendices, the command

\begin{verbatim}
$ pyxis runall
\end{verbatim}

performs the following steps:

\begin{enumerate}
  \item Start python
  \item Find all recipes ({\tt pyxis-*.py}) in the current directory, and interpret them as Python code.
  \item Find all config scripts ({\tt pyxis-*.conf}) in the current directory, and interpret them as Python code. In 
our example, this results in a number of variables being (re)defined.
  \item Look for a Python function named {\tt runall()}, call it, wait for it to finish, and exit.
\end{enumerate}

In this particular case, {\tt runall()} causes a sequence of operations to be performed on a list of measurement
sets (the config file specifies a list of eight MSs), and then averages the resulting images. 
The MSs are processed in sequence.

You can also override variable assignments from the command line. For example:

\begin{verbatim}
$ pyxis MS=other.MS LOG=log.txt imager.npix=2048 cal_ms
\end{verbatim}

would do the following:

\begin{enumerate}
  \item Start python
  \item Load recipes and configs
  \item Set the global MS variable to {\tt other.MS}. 
  \item Set the global LOG variable to {\tt log.txt}, which automatically causes output to be redirected into the named file.
  \item Set the {\tt npix} variable in the {\tt imager} module to 2048.
  \item Call the {\tt cal\_ms()} function, which then executes a calibration step on {\tt other.MS}.
\end{enumerate}

Roughly equivalent usage in the python shell would be:

\begin{verbatim}
  >>> import Pyxis
  # ... bunch of output skipped ...
  >>> v.MS = "other.MS"
  >>> imager.npix = 2048
  >>> cal_ms()
\end{verbatim}

\section{Some Pyxis Highlights}

If you're familiar with Python, you'll have already picked up on some unusual constructs in the example code. 
Most of these are Pyxis features designed to make scripts and recipes easier and more concise (arguably, at the cost of 
``proper'' programming practice...) This section looks at some highlights. 

\subsection{Auto-loading}

Though already mentioned above, this is a key concept that deserves further explanation. When you start up Pyxis 
(whether via the {\tt pyxis} executable, or via an {\tt import Pyxis} statement in your own script or interactive python
session), {\em some of your local files loaded automatically}. Specifically, all the {\tt pyxis-*.py} and {\tt pyxis-*.conf}
files in your local directory are loaded and parsed\footnote{There is also an option to disable this behaviour, see Sect.~\ref{sec:pyxis-special-vars}.}. 

This philosophy here is that your Pyxis processing environment is completely determined by the recipes and configs actually
present in your current working directory. 

\subsection{Automatic logfiles}

Assigning a filename to the global {\tt LOG} variable (whether in a recipe, in a config file, or from the {\tt pyxis} command line) 
automatically causes all further console output to be directed to the named file. It's as simple as that -- but see the comments on 
variable templates below for some wonderful things that can be done with this.

Note that in interactive mode (under ipython), assignments to {\tt LOG} are currently ignored. Conversely, in 
command-line mode, some critical output is duplicated to the console even when a logfile is in use.

\subsection{Invoking shell commands}

The {\tt x} and {\tt xo} objects (``execute'', and ``execute optionally'') provide quick access to shell 
commands:

\begin{verbatim}
  xo.sh("rm -f $IMAGE_LIST");
\end{verbatim}

If the command in question fails (i.e. returns a non-zero error code), Pyxis will either abort the whole script 
(if {\tt x.sh} was used), or log a warning and continue (if {\tt xo.sh} was used, hence the ``optionally'').

\subsection{Variable substitution}

The {\tt "rm -f \$IMAGE\_LIST"} thing above ought to have gotten your attention (and should be familiar to anyone who's 
ever programmed in Perl or shell). This is an example of {\em variable substitution}; all Pyxis functions support variable
substitution on their string arguments as a matter of course. Its equivalent in plain Python would be

\begin{verbatim}
  xo.sh("rm -f "+IMAGE_LIST)
\end{verbatim}

...which doesn't look that much harder, right? However, a typical data reduction recipe will do a great many similar
manipulations with strings (particularly when forming up commands and filenames -- just see the examples in the appendices);
variable substitution just makes things quicker and more concise. A different example really showcases this:

\begin{verbatim}
x.sh("cd ${MS:DIR}; tar zxvf ${MS_TARBALL_DIR>/}${MS:FILE}.tgz")
\end{verbatim}

The {\tt \$\{MS:DIR\}} construct substitutes the directory part of the pathname given by the variable {\tt MS}; the {\tt \$\{MS:FILE\}} construct substitutes the filename part, and {\tt \$\{MS\_TARBALL\_DIR>/\}} substitutes the value of the {\tt MS\_TARBALL\_DIR} variable, followed by a slash, unless {\tt MS\_TARBALL\_DIR} is unset, in which case it substitutes an empty string (and no slash). Doing the same in plain Python is not especially difficult -- but does require more code, and does make the resulting scripts
somewhat harder to look through.

\subsection{``Superglobal'' variables}
\label{sec:superglobals}

In plain Python, variables that are not explicitly scoped\footnote{``{\tt foo.a}'' is explicitly scoped; ``{\tt a}'' is not.} can be either local (only visible within a function) or global (visible to all functions in a particular module). For scripting, it can
also be quite useful to have a special set of ``superglobal'' variables that are shared across all modules (e.g. {\tt MS}, {\tt LSM}, {\tt DDID}, {\tt LOG} in the examples above). In Pyxis, some of these variables have special meaning (e.g. {\tt LOG}). 

In a nutshell, Pyxis handles this as follows. Firstly, all global variables defined in recipes and config files implicitly live in the ``superglobal'' namespace\footnote{Technically speaking, this is the {\tt \_\_main\_\_} module.}. Inside Pyxides modules (see section on writing modules below), these can be accessed via the special {\tt v} (``v'' for ``variable'') object. Assigning to superglobals (even inside of recipes) should be done explicitly via the {\tt v} object, as 

\begin{verbatim}
v.NAME = VALUE  
\end{verbatim}

This ensures that special variables such as {\tt LOG} take effect, and that variable templates (see below) are reinterpreted immediately. Assigning variables on the {\tt pyxis} command line is equivalent to using the {\tt v} object.

\subsection{Variable lists and per-list commands}

Global variables with names that ending in {\tt \_List} can be used to make quick looping constructs. For example, consider 
this bit of the config file:

\begin{verbatim}
v.MS_List = [ "MS/A773_B%d.MS"%b for b in range(8) ];
\end{verbatim}

...which creates a list of filenames. So far, so plain Python. Now, take a look at this bit of code in 
the {\tt runall()} procedure of the recipe, which is called a {\em per-list} construct:

\begin{verbatim}
  per("MS",reset_ms,cal_ms)
\end{verbatim}

...tells Pyxis to do the following: for every value in {\tt MS\_List}, assign it to the {\tt MS} variable, and call
the {\tt reset\_ms} and {\tt cal\_ms} functions. In itself, that may not be so impressive (after all, it's not that different 
from a simple for-loop), but look at the following command-line invocation:

\begin{verbatim}
  $ pyxis -j8 runall
\end{verbatim}

This will process up to 8 measurement sets in parallel (provided you have the cores to handle it). Our 
config file in our case only needs to specify a list of measurement sets to be processed -- Pyxis takes care 
of the rest as soon as it sees a {\tt per()} construct.

It is also possible to execute a per-list construct from the command line. The syntax for the invocation above would be:

\begin{verbatim}
  $ pyxis per[MS,reset_ms,cal_ms]
\end{verbatim}

\subsection{Specifying MS and MS\_List from the command line}

While the per-list construct can be used with any variable, {\tt MS} and {\tt MS\_List} get extra
special treatment due to their popularity in data reduction. Specifically, the {\tt pyxis} executable will treat any 
argument of the form {\tt *.ms} or {\tt *.MS} as an entry in {\tt MS\_List}, and at the same time assign it to the {\tt MS} 
variable. The command

\begin{verbatim}
  $ pyxis A773_B0.MS A773_B1.MS A773_B2.MS per[MS,reset_ms,cal_ms]
\end{verbatim}

will run the per-command over the three MSs specified (overriding the {\tt MS\_List} in the recipe or config). Note also that you can use {\tt per\_ms[commands...]} as an alias for {\tt per[MS,commands,...]}.

Likewise, the command

\begin{verbatim}
  $ pyxis A773_B0.MS reset_ms A773_B1.MS cal_ms
\end{verbatim}

will set {\tt MS=A773\_B0.MS}, run {\tt reset\_ms()}, then set {\tt MS=A773\_B1.MS}, and run {\tt cal\_ms()}.


\subsection{Templated variables}

Templated variables are essentially a variation on variable substitution. For example, this bit of the example config file:

\begin{verbatim}
v.LOG_Template = "${OUTDIR>/}log-${MS:BASE}"
\end{verbatim}

...defines a template for the {\tt LOG} variable. Pyxis keeps track of all template definitions, and re-evaluates 
templates (with variable substitution in effect) whenever some global variable changes. The net result of the above is that
whenever the {\tt MS} variable is changed (typically, when looping over MSs with a per-command), the value of {\tt LOG} changes accordingly, and a new log file is opened. This technique makes it really easy to keep neatly separated outputs.

\subsection{Organizing output files}

Putting some of the above to work together, we have the following in the example config file:

\begin{verbatim}
cal.DESTDIR_Template = '${OUTDIR>/}plots-${MS:BASE}${-stage<STAGE}'
cal.OUTFILE_Template = '${DESTDIR>/}${MS:BASE}${_s<STEP}${_<LABEL}'
\end{verbatim}

The first template specifies a destination directory; the second template specifies a base filename. If you look in the {\tt Pyxides/cal.py} module, you will see that all output files (images, plots, etc.) are named based on the {\tt OUTFILE} variable.
The net result of this is that when looping over MSs, outputs are neatly organized into subdirectories according to the MS name (e.g. {\tt plots-A773\_B0/} for MS {\tt A773\_B0.MS}). In addition, the global {\tt OUTDIR} variable can be used to nest these
subdirectories inside yet another subdirectory. For example,

\begin{verbatim}
  $ pyxis OUTDIR=reduction1 runall
\end{verbatim}

...will cause the outputs associated with {\tt A773\_B0.MS} to go into {\tt reduction1/plots-A773\_B0/}, and the logfile (by virtue of the template assignment above) to go into {\tt reduction1/log-A773\_B0}.

\subsection{Parallel jobs}

The {\tt -jN} option (or {\tt JOBS=N} variable assignment) tells Pyxis that it is allowed to run up to N jobs at once. The jobs are split out when Pyxis encounters a {\tt per()} command. Note that, as with any variable, {\tt JOBS} may be assigned to in the config files as well. A related variable is {\tt JOB\_STAGGER} -- this specifies a delay (in seconds, default being 5 or 10) by which the launching of parallel jobs is staggered. Because Pyxis jobs are similar (same ops -- different data), they will tend to have the same I/O usage pattern; launching jobs simultaneously then causes everybody to access disk together while the CPUs stay idle. Staggering the jobs spreads out the I/O access.

\section{Understanding Pyxis recipes}

Pyxis recipes are meant to be modified (and ultimately generated) by you, the end-user! This section therefore provides a more in-depth looks at what recipes are and how they're written.

A Pyxis recipe is, in the end, just a Python script for doing something or other. A minimalistic recipe could consist of just one Python function with a name like {\tt run()}. When you start Pyxis, the recipe is automatically loaded (provided it's in the current directory), and the {\tt run()} function becomes available. It can then be invoked from the command line as

\begin{verbatim}
  $ pyxis run
\end{verbatim}

or interactively, as

\begin{verbatim}
  $ ipython
  In [1]: import Pyxis
  In [2]: run()
\end{verbatim}

Note that in both cases, you must be in the directory containing the Pyxis recipe file for this to work!

A real-life recipe, as in the example of Appendix~\ref{app:recipe}, defines functions for a number of simple tasks ({\tt reset\_ms()}, {\tt cal\_ms()}, and perhaps one or more ``super-task'' functions that call several tasks in sequence ({\tt runall()}). 

If you've running Pyxis interactively, the {\tt printvars()} function will provide a summary of currently defined functions (among other things).


\subsection{Passing parameters}

Users of casapy and pybdsm will be familiar with a task-based interface, where one has the concept of a ``current task'', and sets its parameters by assigning values on the command line:

\begin{verbatim}
 >>> filename = 'foo.fits'  # sets parameter of task
 >>> go                     # runs the current task
\end{verbatim}

Contrast this to an argument-based interface:

\begin{verbatim}
 >>> process_image(filename='foo.fits')
\end{verbatim}

The first approach simplifies interactive use; the second approach makes for cleaner programming and more flexible interfaces. Since scripting occupies somewhat of a middle ground, Pyxis encourages one to use a mixed approach. Pyxis ``tasks'' are nothing but Python functions, and must be called as functions with arguments. Typically, many (or all) of their parameters (e.g. current MS, current LSM) can be specified by setting global variables, which gives the overall interface a more task-based feel. Functions at the recipe level (at least in my recipes) tend to rely on globals almost exclusively (see {\tt reset\_ms()} and {\tt cal\_ms()} in the example), while functions within Pyxides (e.g. {\tt cal.stefcal()}) use proper arguments. However, many of these arguments have default values that are defined in terms of globals, and so may be omitted from the call. For example, here's the documentation for the {\tt cal.stefcal()} function (note that this documentation may be accessed from an interactive session by calling {\tt help(cal.stefcal)}, or in ipython as simply ``{\tt cal.stefcal?}''):

{\scriptsize
\begin{verbatim}
>>> import Pyxis  # remember to do this in the example directory!
...
>>> help(cal.stefcal)

  Help on function stefcal in module cal:

stefcal(msname='$MS', section='$STEFCAL_SECTION', label='G', apply_only=False, diffgains=None, 
    flag_threshold=None, output='CORR_RES', plotvis='$PLOTVIS', 
    dirty=True, restore=False, restore_lsm=True, args=[], options={}, **kws)

    Generic function to run a stefcal job.
      
      'section'         TDL config file section
      'label'           will be assigned to the global LABEL for purposes of file naming
      'apply_only'      if true, will only apply saved solutions
      'diffgains'       set to a source subset string to solve for diffgains. Set to True to use "=dE"
      'flag_threshold'  threshold flaging post-solutions. Give one threshold to flag with --above,
                        or T1,T2 for --above T1 --fm-above T2
      'output'          output visibilities ('CORR_DATA','CORR_RES', 'RES' are useful)
      'plotvis'         if not empty, specifies which output visibilities to plot using plot-ms (see plot.ms.py --help) 
      'dirty','restore' image output visibilities (passed to make_image above as is)
      'args','options'  passed to the stefcal job as is (as a list of arguments and kw=value pairs), 
                        can be used to supply extra TDL options
      extra keywords:   passed to the stefcal job as kw=value, can be used to supply extra TDL options, thus
                        overriding settings in the TDL config file. Useful arguments of this kind are e.g.:
                        stefcal_reset_all=True to remove prior gains solutions.
      
    The following variables also apply:
    
      mqt.TDLCONFIG        default TDL config file
      cal.STEFCAL_DIFFGAINS current file for diffgain solutions
      ...
      v.MS                 current measurement set
      ...
\end{verbatim}
}

The above means the following: the first argument to {\tt stefcal} is an MS name, and the second is a TDL config section (see Sect.~\ref{sec:tdlconfig}); if not given, these default to the global {\tt MS} and {\tt STEFCAL\_SECTION} variables. Note that the former is a ``superglobal'' variable shared across modules (see Sect.~\ref{sec:superglobals}, \ref{sec:variables}), while the latter is a variable specific to the {\tt cal} module, as the documentation clearly indicates. The {\tt stefcal} function can therefore be equivalently invoked as either

\begin{verbatim}
>>> cal.stefcal("my.MS","myconfig_section") 
\end{verbatim}

or

\begin{verbatim}
>>> v.MS = "my.MS"
>>> cal.STEFCAL_SECTION = "myconfig_section"
>>> cal.stefcal() 
\end{verbatim}


\subsection{Configuration and global variables}
\label{sec:variables}

Pyxis configuration is simple in principle, as it is entirely based on Python global variables. Python imposes a separate
``global scope'' on each module it loads: that is, each module has its own set of globals. This makes for good programming practice, but for scripting applications, one usually needs an easy way to share variables across modules. Pyxis provides such a way via a
common ``superglobals'' scope, which may be accessed via the built-in {\tt v} object:

\begin{verbatim}
# sets global variable in the cal module
>>> cal.STEFCAL_SECTION = "mysection"   
# sets "superglobal" variable
>>> v.MS = "my.ms" 
\end{verbatim}

Within recipes and interactive sessions (as opposed to within modules), the global scope and the superglobals scope is one and the same\footnote{Technically speaking, the superglobals scope is the same thing as the {\tt \_\_main\_\_} namespace, and all recipes are
executed in this namespace.}. However, proper superglobals require special treatment, in that they must be assigned to via the {\tt v} object:

\begin{verbatim}
>>> v.MS = "my.ms" 
>>> print MS,v.MS,MS is v.MS   
my.ms my.MS True
>>>
\end{verbatim}

A simple assignment statement like

\begin{verbatim}
>>> MS = "my.ms" 
\end{verbatim}

is technically correct, but breaks the updating of templates and other Pyxis internals (see below), so it is best avoided where
shared variables defined by Pyxides modules ({\tt MS}, {\tt LSM}, etc.) are concerned. Ordinary variables (i.e. specific to a recipe, 
and not shared across modules) do not need special treatment.

Given that Pyxis loads things in a specific order, that order imposes a logical hierarchy on configuration (i.e. variable assignments), from the most generic to the most specific:

\begin{itemize}
  \item All recipes ({\tt pyxis-*.py}) in the current directory are loaded and parsed in alphabetical order. 
  \item A recipe will usually import one or more Pyxides modules first thing. Pyxides modules define their own globals (and some superglobals), and initialize them with default values.
  \item Subsequent statements in the recipe can then change the values of superglobals and module-level globals (see example in Appendix~\ref{app:recipe}).
  \item Once all recipes have been loaded, Pyxis will load and parse all config files ({\tt pyxis-*.conf}), also in alphabetical 
  order. Config files will typically change even more variables.
  \item Finally, command-line arguments to the {\tt pyxis} utility can be used to refine the configuration even further:
\begin{verbatim}
$ pyxis MS=my.ms cal.STEFCAL_SECTION=mysection runall
\end{verbatim}
  \item Alternatively, variables can also be changed in an interactive session (see above).\
\end{itemize}

\subsubsection{MeqTrees scripts and TDL configs}

Python variables are not the whole story, since external tools called by recipes and modules can have their own specialized configuration files. In particular this concerns MeqTrees. A MeqTrees script typically needs at least two files: a TDL script (e.g. {\tt calico-stefcal.py}), and a TDL config file (typically, {\tt tdlconf.profiles}). The four steps to running a MeqTrees script are:

\begin{enumerate}
  \item Load TDL options from a specific {\em section} of the TDL config file.
  \item Override some of the loaded options, if necessary.
  \item Compile the TDL {\em script} with these options, resulting in some sort of specialized tree being set up.
  \item Run a TDL {\em job} defined by the script.
\end{enumerate}

A general interface to this is provided by {\tt mqt.run()} in the {\tt mqt} Pyxides module. More specialized functions such as {\tt cal.stefcal()} wrap this in an application-specific interface. Behind the scenes, such functions implement logic of their own at step 2 of the above, by overriding some TDL options in accordance with their input parameters. Normally this behaviour is hidden from view, but a typical interface pattern is to use of keyword arguments and global variables to override even more TDL options. Take as an example this line in our example recipe:

\begin{verbatim}
  cal.stefcal(stefcal_reset_all=True,output="CORR_RES",restore=False,flag_threshold=(1,.5))
\end{verbatim}

Here, the {\tt stefcal\_reset\_all} keyword argument is passed from {\tt cal.stefcal()} directly to the underlying MeqTrees script, thus overriding whatever setting is specified in the {\tt tdlconf.profiles} file. Likewise, the statement

\begin{verbatim}
cal.STEFCAL_TDLOPTS = "calibrate_ifrs=-<144 stefcal_verbose=3"
\end{verbatim}

also causes the specified TDL options to be overridden when {\tt cal.stefcal()} invokes MeqTrees.

In practice, this means that you need to be aware of a few things:

\begin{itemize}
\item MeqTrees scripts and TDL configs should be bundled with the Pyxis recipes that use them, as in the {\tt example} directory here.
\item TDL configs can contain some ``deep'' options of the script that a recipe does not change directly, but na\"ive users should not normally need to worry about this.
\item Power users can always manipulate TDL configs by using the meqbrowser, and loading/running scripts interactively.
\end{itemize}

\subsection{Variable substitution}
\label{sec:varsubst}

\subsection{Variable templates}

\subsection{Special variables}

\subsection{Built-in Pyxis functions}


\section{Pyxides modules}


\appendix\newpage

\section{Example Pyxis recipe}
\label{app:recipe}

A copy of this file can be found under {\tt Pyxis/doc/example/pyxis-wsrt21cm.py}.

{\scriptsize
\verbatiminput{example/pyxis-wsrt21cm.py}
}
\section{Example Pyxis config}
\label{app:config}

A copy of this file can be found under {\tt Pyxis/doc/example/pyxis-a773.conf}.

{\scriptsize
\verbatiminput{example/pyxis-a773.conf}
}



\end{document}
